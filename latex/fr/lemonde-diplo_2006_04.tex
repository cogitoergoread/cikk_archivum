% -*- mode: LaTeX; coding: utf-8 -*-
\documentclass[a4paper,twocolumn]{article}
\usepackage[T1]{fontenc}
\usepackage[latin2]{inputenc}
\usepackage[magyar]{babel}
\begin{document}
\author{Guy Scarpetta}
\title{  Pasolini, un réfractaire exemplaire }
\date{2006, április 13 \thanks{LE MONDE DIPLOMATIQUE | février 2006 | Page 24  } }
\topmargin -1.8cm        % read Lamport p.163
\oddsidemargin -1.8cm   % read Lamport p.163
\evensidemargin -1.8cm  % same as oddsidemargin but for left-hand pages
\textwidth      7.5in
\textheight	10.0in
\maketitle



% \textqq{ }

L’anniversaire de l’assassinat de Pier Paolo Pasolini (1922-1975) a fourni l’occasion d’un important retour sur la vie et l’œuvre de cet écrivain et cinéaste italien. Premier constat : Pasolini fait partie de ces intellectuels réfractaires à toute tentative de récupération par la culture dominante. Et, second constat : en ces temps d’avachissement idéologique, son anticonformisme subversif demeure intensément réconfortant.

 

Par Guy Scarpetta
Ecrivain. Auteur notamment de L’Age d’or du roman (Grasset, Paris, 1996), de Pour le plaisir (Gallimard, Paris, 1998), de Variations sur l’érotisme (Descartes et Cie, Paris, 2004).




		

Le 2 novembre 1975, Pier Paolo Pasolini était sauvagement assassiné, sur un terrain vague, près d’Ostie. A l’occasion du trentième anniversaire de cette mort, de nombreuses publications ont vu le jour, notamment en France  (1), qui témoignent de la fascination que l’écrivain et cinéaste italien continue à exercer. Les circonstances mêmes du crime, encore incomplètement élucidées  (2), n’ont pu que contribuer à développer, à son égard, une véritable légende noire – dont émane une image proprement mythologique, celle de l’ange du mal, de l’hérétique persécuté, du dernier grand artiste maudit. Mais sans doute le temps est-il venu, malgré tout, de dépasser cette image – et de voir, plutôt, en Pasolini, à travers l’exceptionnelle diversité des registres qu’il a traversés (poésie, roman, cinéma, essais critiques et théoriques, interventions journalistiques), un formidable exemple, vivant, paradoxal, singulier, d’intellectuel engagé.

Encore faut-il s’entendre sur ce terme, recouvert par des tonnes de rouille, et largement discrédité, aujourd’hui, par tous les défenseurs, plus ou moins masqués, de l’ordre établi. Pasolini, d’évidence, n’était ni un « intellectuel de parti » (docile, chargé d’appliquer la ligne), ni un « intellectuel organique » au sens d’Antonio Gramsci (chargé de contribuer à l’hégémonie culturelle du « bloc historique » prétendant au pouvoir), ni même un écrivain engagé selon le modèle sartrien (détenteur du sens de l’histoire, et subordonnant toute pratique d’expression aux exigences d’un combat collectif). Plutôt quelqu’un pour qui la tâche d’un artiste ou d’un intellectuel, dès lors qu’il se veut solidaire des « damnés de la terre », est de mettre en crise et de subvertir les conceptions du monde dominantes, d’explorer le non-dit des représentations convenues (y compris, le cas échéant, celles de son propre camp), de faire surgir le refoulé du consensus social et culturel – sans rien céder, jamais, sur sa singularité (ce que Juan Goytisolo, aujourd’hui, nomme l’« intellectuel sans mandat  » (3)).

C’est ainsi que Pasolini, s’il ne renia jamais complètement l’engagement communiste de ses années de formation, n’en éprouva pas moins le besoin constant de surmonter, ou d’excéder, ce qu’il nommait le « conformisme des progressistes ». Ce qui explique, par exemple, à l’époque où le communisme officiel, institutionnel, misait surtout sur le prolétariat organisé des villes industrielles, son insistance sur le monde paysan (avec ses codes, ses valeurs spécifiques), ou sur le sous-prolétariat des banlieues urbaines (façon de résister aux prétendus impératifs de l’histoire, de se focaliser sur ce qu’elle tend à marginaliser ou à exclure). Ou encore son intérêt pour le tiers-monde (dans lequel, selon lui, « il y a déjà quelques formes de prise de conscience qui contredisent à la fois le rationalisme marxiste et le rationalisme bourgeois »), ou pour certains mouvements de la gauche radicale américaine, comme les Black Panthers, crédités de « jeter leur corps dans la lutte », de déborder les schémas révolutionnaires classiques.

Ce marxisme hétérodoxe est aussi au cœur de l’engagement culturel et artistique de Pasolini. Très vite, celui-ci a compris que la culture progressiste de l’après-guerre, issue du combat antifasciste, a désormais épuisé ses fonctions (« Le temps de Brecht et de Rossellini est révolu »). Mais il ne s’agit pas pour autant de céder au purisme et au formalisme des avant-gardes littéraires des années 1960 (en Italie, par exemple, les poètes du Groupe 63), à qui il reproche de mener une lutte abstraite, inoffensive, « purement linguistique », d’être prisonniers d’un mode de vie « petit-bourgeois », et de masquer derrière leurs proclamations antinaturalistes une pure et simple « terreur à l’égard de la réalité ». Point-clé : l’engagement, pour Pasolini, surgit aussi de l’expérience directe, de la façon de vivre, de l’implication subjective et physique, dans la réalité (proximité, ici, avec quelqu’un comme Jean Genet). Et cette implication, c’est ce qui passe tout autant dans sa poésie, lyrique, ambiguë, scandaleuse, dans ses romans, ou dans son art du cinéma.

Car l’intérêt du cinéma, pour lui, est d’être une écriture directement en prise sur le réel, une façon de capter et de révéler la réalité comme un langage (donc de la dénaturaliser) – découpant et isolant des plans (d’où son caractère explicitement « fétichiste ») dans le grand « plan-séquence ininterrompu de la vie ». Ce dont procède, en définitive, l’une des œuvres cinématographiques les plus bouleversantes et les plus audacieuses du XXe siècle : non seulement un authentique cinéma d’auteur (ou ce qu’il désignait, pour se démarquer des normes narratives du cinéma commercial courant, comme un « cinéma de poésie »), mais encore un art éminemment paradoxal, à la fois primitif et maniériste, à la fois réaliste (dans son amour concret, son aptitude à faire percevoir le « langage des corps ») et hypercultivé (dans sa façon de convoquer et de mêler, au second degré, des éléments issus de la peinture ancienne, de la musique classique ou populaire, de la littérature, en une superbe impureté).

Et cela, qu’il s’agisse de réintroduire la tragédie dans le monde du sous-prolétariat (Accattone, Mamma Roma), de ressusciter les mythes d’une Grèce barbare, préclassique (Œdipe roi, Médée), de restituer au récit christique sa violence et sa portée subversive (L’Evangile selon Matthieu), d’élaborer d’étranges paraboles, où la grâce s’enchevêtre à l’obscénité, pour déstabiliser le conformisme ambiant (Théorème, Uccellacci e uccellini, Porcherie), d’explorer le dehors de la culture bourgeoise, ses antécédents populaires occultés (Le Décaméron, Les Contes de Canterbury) ou son altérité orientale (Les Mille et Une Nuits), ou de propulser la noirceur sadienne dans le contexte du fascisme agonisant (Salo ou les Cent Vingt Journées de Sodome). Autant de films qui continuent, plus de trente ans après, de nous troubler, par leur énigmatique beauté – et qui ne peuvent qu’accuser, par contraste, l’état actuel du cinéma, majoritairement soumis à la débilité marchande de l’industrie du divertissement (une telle œuvre, aujourd’hui, n’aurait tout bonnement aucune chance d’exister).

Pasolini a-t-il été réactionnaire ? Soutenir cela, comme on le fait parfois, est un contresens parfait. Ce qui est vrai, c’est qu’il a pu parfois soutenir des opinions « indéfendables », à l’opposé de ce qui se présentait comme moderne ou progressiste (à propos des mouvements étudiants de 1968, par exemple, ou du débat des années 1970 sur l’avortement). Mais, à relire aujourd’hui ces interventions polémiques, on s’aperçoit qu’elles visaient avant tout à provoquer les intellectuels de la gauche conformiste (y compris ceux qui étaient ses amis : Alberto Moravia, Italo Calvino, Umberto Eco) – et à les amener à trahir, dans leurs réactions, justement, ce que leur « progressisme » apparent pouvait avoir de fondamentalement bien-pensant.

Plus généralement, il est certain que Pasolini, qui idolâtrait Rimbaud, n’a jamais cru pour autant qu’il fallait être « absolument moderne ». Qu’il n’a jamais considéré la nostalgie, même largement imaginaire (nostalgie de la Nature, du Maternel, de l’Innocence perdue), comme une façon de s’opposer à un monde où la modernité peut parfaitement s’identifier à la barbarie. En ce sens, ce qu’il allait chercher dans la nostalgie du Frioul, du monde rural, d’une diversité culturelle et dialectale menacée par le « progrès », ou dans celle des cultures prébourgeoises (Boccace, Chaucer) et extra-occidentales (Les Mille et Une Nuits), n’était pas très différent de ce qui l’attirait dans le tiers-monde, ou dans le sous-prolétariat des borgate romaines : une manière de s’appuyer sur les « forces du passé » pour mieux combattre le présent lorsque celui-ci devient destructeur.

Passage, si l’on veut, d’une position progressiste (adhésion aveugle à la modernité, remplacement de l’ancien par le nouveau) à une position de résistance (incluant la résistance au nouveau, lorsque celui-ci est synonyme d’oppression supplémentaire, de conformisme, d’uniformité). Le coup de génie de Pasolini (qui le distingue radicalement, soit dit en passant, de tous les « néo-réacs » d’aujourd’hui), c’est précisément d’avoir su transformer la nostalgie en force critique. Inutile d’insister, je crois, sur ce qu’une telle attitude, isolée à son époque, peut avoir aujourd’hui d’étonnamment actuel : aujourd’hui, c’est-à-dire dans une situation où les pires régressions (notamment sociales) se présentent comme des « modernisations » (c’est la rhétorique même de la vulgate libérale) – et où il peut être révolutionnaire, du coup, de contester le type de « modernité » imposé par la tyrannie du marché...

Il est un dernier point, enfin, où l’engagement de Pasolini apparaît comme prodigieusement anticipateur – et presque prophétique. C’est celui, qu’il est pratiquement seul à repérer à son époque  (4), qui concerne la véritable « mutation anthropologique » s’opérant sous ses yeux, par laquelle la bourgeoisie au pouvoir étend et renforce sa domination. Pasolini avait, dans sa Trilogie de la vie, chanté la liberté sexuelle (déculpabilisée) d’un monde populaire non encore asservi au puritanisme bourgeois.

Or, dès la sortie de ces trois films, il éprouve la nécessité de les « abjurer » : précisément parce qu’il se rend compte que le pouvoir des années 1970 peut parfaitement accepter la « libération sexuelle », et promouvoir en ce domaine la permissivité, dès lors que chacun est assigné à un rôle de consommateur, et que le sexe devient une marchandise comme les autres. C’est ainsi que le sexe cesse d’être une valeur de scandale (puisque le puritanisme disparaît) : il est à son tour absorbé, intégré, il n’est plus tabou (donc plus sacré : la marchandisation de toutes les activités humaines est une « profanation »), il relève désormais du nouveau conformisme de la consommation.

Pasolini fut sensible à cela, bien sûr, à partir de sa propre homosexualité, dont il redoute la dissolution dans la norme (« Il est intolérable, écrit-il, d’être toléré ») – et qui valait pour lui, d’évidence, beaucoup plus comme un défi que comme un facteur d’appartenance : « Ce n’est pas tant l’homosexuel qu’ils ont toujours condamné que l’écrivain sur qui l’homosexualité n’a pas eu de prise comme moyen de pression, de chantage à rentrer dans le rang  (5). » Mais ce qui est le plus important, c’est le constat plus vaste qu’il opère à partir de là : il existe désormais un pouvoir à la fois économique et médiatique (les maîtres du monde sont aussi ceux de sa représentation), dont l’horizon est d’imposer le règne du troupeau généralisé, de la middle class planétaire, désacralisatrice et uniformisatrice.

A cela, comme toujours chez lui, une perception d’abord physique : les sous-prolétaires des borgate se sont mis à rêver d’entrer dans la norme, à avoir honte de leurs codes anciens, à répudier leur culture spécifique, ils commencent à ressembler aux étudiants issus de la bourgeoisie (ils ont les mêmes comportements, les mêmes jeans, les mêmes cheveux longs, presque le même langage) ; le tiers-monde commence lui aussi à se mouler dans la pseudo-universalité de l’Occident technicien et consommateur, à commencer par le tiers-monde interne à l’Italie (le Mezzogiorno) ; le « Centre », en particulier grâce à ce terrifiant instrument d’homologation et de normalisation qu’est la télévision (qui devient pour lui l’ennemi principal, dont il prône la « destruction »), impose un modèle unique, exclusif – c’est, dit-il, le « nivellement brutalement totalitaire du monde », l’« ordre dégradant de la horde ». En somme, ce que le fascisme historique avait échoué à réaliser, le nouveau pouvoir conjugué du marché et des médias l’opère en douceur (dans la servitude volontaire) : un véritable « génocide culturel », où le peuple disparaît dans une masse indifférenciée de consommateurs soumis et aliénés.

Le constat est sombre, déchirant – il n’en est pas inexact pour autant : tout cela, depuis trente ans, n’a guère fait que s’accentuer. Face à quoi la résistance, pour Pasolini, se doit d’être subjective autant que politique. Pas d’autre façon de contester cet « ordre » que d’affirmer farouchement sa singularité, son écart, son irréductibilité (seule énergie que le marché et le spectacle sont impuissants à assimiler). Leçon plus que jamais actuelle, aux antipodes de ce « conformisme de la rébellion » qui s’épanouit dans le monde intellectuel, et qui est le meilleur complice de l’ordre établi.

(1) Outre plusieurs rééditions de textes de Pasolini en livre de poche, on peut citer, récemment : René de Ceccatty, Pasolini, Gallimard, coll. « Folio Biographies », Paris, 2005 ; René de Ceccatty, Sur Pier Paolo Pasolini, du Rocher, Paris, 2005 ; La Longue Route de sable, texte inédit de Pasolini, Arléa, Paris, 2004 ; Bertrand Levergeois, Pasolini. L’alphabet du refus, Editions du Félin, Paris, 2005 ; Marco Tullio Giordana, Pasolini, mort d’un poète, Seuil, Paris, 2005.

(2) Voir Marco Tullio Giordana, op. cit.

(3) A propos de Günter Grass, dans Juan Goytisolo, Cogitus interruptus, Fayard, Paris, 2001.

(4) A l’exception, notable, de Guy Debord et des situationnistes – que Pasolini ne connaissait guère.

(5) On peut imaginer le rire de Pasolini s’il savait qu’il est aujourd’hui l’un des objets fétiches des Gays and Lesbian Studies – lui pour qui la sexualité (homo ou hétéro) était avant tout un phénomène singulier, irréductible à toute commune mesure (« L’érotisme, disait-il, est un phénomène excessivement individuel » ; « Il y a des gouffres entre ceux qui appartiennent à la même famille érotique ») et qui était, par là même, étranger à toute fierté d’appartenance (gay pride).

\end{document}
